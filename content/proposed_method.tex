\frame{
    \frametitle{Retention times (RTs) for metabolite identification} 
    
\begin{itemize}
     \item State-of-the-art machine learning metabolite identification methods use \emph{only} MS/MS information \cite{Brouard2016,Duehrkop2015} 
     \item Retention times are \emph{valuable} orthogonal information \cite{Ruttkies2016,Stanstrup2015,Aicheler2015} \hfill\ex{distinction of diastereoisomers}
\end{itemize}

\begin{alertblock}{Challenges utilizing RTs}
\begin{enumerate}
    \item Measurements are \emph{LC-system specific}. 
    \item Public datasets are relatively \emph{small} and originate from \emph{heterogeneous systems}
\end{enumerate}
\end{alertblock}
}

\frame{
    \frametitle{Proposed method to tackle the challenges}
    
\begin{enumerate}
    \item<1> \textbf{Predict the pairwise retention order} of molecular candidate structures using preference learning
    \item<1>[$\circ$] Retention orders are largely preserved across LC-systems \cite{Stanstrup2015}.
    \item<1>[$\circ$] Prediction model can be trained on \emph{multiple} retention time datasets arising from \emph{heterogeneous} LC-systems. 
    \item<2> \textbf{Integrating predicted candidate retention orders and MS/MS based scores} to \emph{jointly} identify a set of metabolites.
\end{enumerate}   
}
